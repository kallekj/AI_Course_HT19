\documentclass{article}
\usepackage[utf8]{inputenc}
\usepackage{booktabs}
\usepackage{float}
\usepackage[a4paper, width=150mm, top=25mm, bottom=25mm, bindingoffset=6mm]{geometry} 
\usepackage{verbatim}


\title{AI Lab 2}
\author{Karl-Johan Djervbrant}

% Hide section numbering
\makeatletter
\def\@seccntformat#1{%
  \expandafter\ifx\csname c@#1\endcsname\c@section\else
  \csname the#1\endcsname\quad
  \fi}
\makeatother

\begin{document}

    \maketitle

    \section{Task 1}
      \subsection{a)}
      \begin{table}[h]
        \label{tbl:task1A}
        \centering
        \begin{tabular}{lcccc}
          \toprule
          {Classifiers $k=5$} & Algorithm & Accuracy (\%) & Correct &   Time (s) \\
          \midrule
          Own KNN\_Classifier\_fast &  euclidean &       0.8525 &     341/400 &    4.76143 \\
          Own KNN\_Classifier      &  euclidean &       0.8525 &     341/400 &    54.7676 \\
          Sklearn                &  euclidean &       0.8525 &     341/400 &  0.0145493 \\          
          \bottomrule
        \end{tabular}
        \caption{Performance of own KNN algorithm vs sklearn implementation.}
      \end{table}
      In this task we had to implement our own version of a KNN\_Classifier. The Algorithm works by
      calculating the distance from one data point to another, and assigning the class to the new data point 
      which the K-Nerest-Neighbors has.
      
      The difference between \verb$KNN_Classifier$ and \verb$KNN_Classifier_fast$ is that the faster
      version only keep track of the K closest distances currently calculated, this means that you 
      don't have to sort the whole list of calculated distances, (which increases every time you calculate a distance).
      With the faster algorithm, you only need to sort a list with K elements in it. This also saves on memory allocation.

      As we can see in Table \ref{tbl:task1A}, both version of the own implementation perform equally good compared to the 
      KNN-algorithm implemented by scikit-learn. The only difference is the execution time, where the scikit-learn 
      implementation is much faster.


      \subsection{b)}
        \begin{table}[h]
          \centering
          \label{tbl:task1b}
          \begin{tabular}{lrr}
            \toprule
            {Classifiers} &   Score ($R^2$)&  Time (s) \\
            \midrule
            SVC, $kernel=linear,~C=0.025$ &  0.9475 &  0.230703 \\
            Decision Tree, $max\_depth=5$ &  0.8800 &  0.005497 \\
            Random Forest, $max\_depth=5, n\_esti=10, max\_feat=1$ &  0.9100 &  0.013979 \\
            \bottomrule
          \end{tabular}
          \caption{Performance between three classification algorithms, higher score is better.}
        \end{table}
        In this task we got to implement and compare three different algorithms from the scikit-learn 
        library. I choose to test the performance on Scalar-Vector-Classifier (SVC), Decision Tree Classifier and
        Random Forest Classifier.
        From the data in Table \ref{tbl:task1b} we can see that they perform good on different measures, the SVC algorithm 
        does the best prediction but is also the slowest. The Decision Tree algorithm is the worst one, but also the fastest.
        

      \subsection{c)}
        \label{ssc:1c}
        \begin{table}[h]
          \centering
          \label{tbl:task1c}
          \begin{tabular}{cccc}
            \toprule
            Classifier &  Neighbors & Distance Measure &           Score (Mean/Std)\\
            \midrule
            KNN &          3 &              Manhattan &  0.73 (+/-0.45) \\
            KNN &          5 &              Manhattan &  0.73 (+/-0.45) \\
            KNN &          7 &              Manhattan &  0.74 (+/-0.45) \\
            KNN &          9 &              Manhattan &  0.71 (+/-0.47) \\
            KNN &         11 &              Manhattan &  0.73 (+/-0.48) \\
            KNN &          3 &              Euclidean &  0.70 (+/-0.45) \\
            KNN &          5 &              Euclidean &  0.71 (+/-0.48) \\
            KNN &          7 &              Euclidean &  0.70 (+/-0.50) \\
            KNN &          9 &              Euclidean &  0.68 (+/-0.48) \\
            KNN &         11 &              Euclidean &  0.67 (+/-0.48) \\
            KNN &          3 &  Minkowski p=$2^{1.5}$ &  0.68 (+/-0.47) \\
            KNN &          5 &  Minkowski p=$2^{1.5}$ &  0.70 (+/-0.49) \\
            KNN &          7 &  Minkowski p=$2^{1.5}$ &  0.70 (+/-0.47) \\
            KNN &          9 &  Minkowski p=$2^{1.5}$ &  0.67 (+/-0.48) \\
            KNN &         11 &  Minkowski p=$2^{1.5}$ &  0.67 (+/-0.47) \\
            \bottomrule
          \end{tabular}
          \caption{Cross-Validation score of KNN-Classifier with different amount of neighbors and distance algorithms}
        \end{table}
        To change the distance algorithm used, you initialize the constructor with the parameter $p$ set to a 
        specific value. $p=1$ is for Manhattan-distance and $p=2$ is for Euclidean-distance.
        If you set $p$ to another value you can calculate other types of distances, the algorithm used 
        is the Minkowski-distance algorithm.
        In Table \ref{tbl:task1c} we can se the difference in performance (using cross-validation) depending on distance algorithm 
        and amount of neighbors. On this data set the Manhattan-distance perform slightly 
        better than the Euclidean-distance, the best one is with $K=7$.
        \subsection{e)}
          \begin{table}[h]
            \centering
            \label{tbl:task1e}
            \begin{tabular}{llll}
              \toprule
              {Regressors $k=5$} &  Algorithm &     Score ($R^2$)&    Time (s) \\
              \midrule
              Own KNN\_Regressor\_fast &  euclidean &  0.395646 &     9.06785 \\
              Own KNN\_Regressor      &  euclidean &  0.395646 &     78.4536 \\
              Sklearn                &  euclidean &  0.395646 &  0.00223327 \\
              \bottomrule
            \end{tabular}
            \caption{Performance of own KNN-Regression algorithm vs sklearn on estimating Fuel Consumption.}
          \end{table}
          A regression algorithm is used to estimate a value for a specific feature, in this case we use it to estimate the fuel consumption. 
          The only modification done to the algorithm compared to the classification version is that it now returns the mean of the K nearest classes. 
          As we can see in the Table \ref{tbl:task1e}, KNN-Regression isn't working too well on estimating the
          fuel consumption on this data set. This is partly due to that KNN is sort of a improved look-up table 
          and is relatively simple, which is why it doesn't perform great in all occasions.
        \subsection{f)}
          \begin{table}[h]
            \centering
            \label{tbl:task1f}
            \begin{tabular}{lrr}
              \toprule
              {Regressors} &     Score ($R^2$) &  Time (s) \\
              \midrule
              SVR,  $kernel=linear,~C=0.025$    &  0.768634 &  3.850483 \\
              Decision Tree Regressor, $max\_depth=5$ &  0.754809 &  0.007603 \\
              Random Forest Regressor, $max\_depth=5, n\_esti=10, max\_feat=1$ &  0.608807 &  0.023281 \\
              \bottomrule
            \end{tabular}
            \caption{Performance of three different regression algorithms from the sklearn library.}
          \end{table}
          With the use of other regression algorithms we can get much better prediction results compared to KNN-Regression, as we can see in Table \ref{tbl:task1f} SVR and Decision Tree Regressor 
          perform almost twice as good as KNN-Regression. Out the tree algorithms tested, the SVR (Scalar Vector Regressor) performed best, bust was also much slower compared to Decision Tree Regressor. 
          The DTR algorithm did almost score as high as the SVR algorithm but were much faster to calculate.

        \subsection{g)}
          \begin{table}[h]
            \centering
            \label{tbl:task1g}  
            \begin{tabular}{cccc}
              \toprule
              Classifier &  Neighbors &    Distance Measure &    Score (Mean/Std)\\
              \midrule
              KNN &          3 &              Manhattan &  -0.31 (+/-2.87) \\
              KNN &          5 &              Manhattan &  -0.27 (+/-2.94) \\
              KNN &          7 &              Manhattan &  -0.17 (+/-2.61) \\
              KNN &          9 &              Manhattan &  -0.15 (+/-2.46) \\
              KNN &         11 &              Manhattan &  -0.11 (+/-2.38) \\
              KNN &          3 &              Euclidean &  -0.46 (+/-3.36) \\
              KNN &          5 &              Euclidean &  -0.35 (+/-3.02) \\
              KNN &          7 &              Euclidean &  -0.25 (+/-2.65) \\
              KNN &          9 &              Euclidean &  -0.23 (+/-2.61) \\
              KNN &         11 &              Euclidean &  -0.22 (+/-2.55) \\
              KNN &          3 &  Minkowski p=$2^{1.5}$ &  -0.50 (+/-3.33) \\
              KNN &          5 &  Minkowski p=$2^{1.5}$ &  -0.35 (+/-3.07) \\
              KNN &          7 &  Minkowski p=$2^{1.5}$ &  -0.26 (+/-2.77) \\
              KNN &          9 &  Minkowski p=$2^{1.5}$ &  -0.25 (+/-2.70) \\
              KNN &         11 &  Minkowski p=$2^{1.5}$ &  -0.26 (+/-2.74) \\
              \bottomrule
            \end{tabular}
            \caption{Cross-Validation score of KNN-Regression with different amount of neighbors and distance algorithms}
            As previously discussed in Task \ref{ssc:1c} 1c), the distance algorithm used is changed by changing the value of $p$.
            In Table \ref{tbl:task1g} we see that KNN-Regression perform quite bad on estimating the Fuel Consumption, with a low score and 
            high standard deviation, but the best performing one out of the tested values were with Manhattan Distance as distance measure and $K=11$.
          \end{table}
      \section{Task 2}
          \subsection{a)}
            To use the data in \verb$Lab4PokerData.csv$ we first need to translate the data to something a classification algorithm can understand, the text in the data
            were translated as followed:
            \begin{table}[H]
              \centering
              \label{tbl:task1f}
              \begin{tabular}{lrrr}
                \toprule
                Values &     Ranks &  Actions & Empty Cell \\
                \midrule
                2: 2 & 0: highcard & 0: Fold & -1: NO\_DATA \\
                3: 3 & 1: onepair & 1: Call & {} \\
                4: 4 & 2: twopair & 2: Raise & {} \\
                5: 5 & 3: 3ofakind & 3: Allin & {} \\
                6: 6 & 4: straight & {} & {} \\
                7: 7 & 5: flush & {} & {} \\
                8: 8 & 6: fullhouse & {} & {} \\
                9: 9 & 7: 4ofakind & {} & {} \\
                T: 10 & 8: straightflush & {} & {} \\
                J: 11 & {} & {} & {} \\
                Q: 12 & {} & {} & {} \\
                K: 13 & {} & {} & {} \\
                A: 14 & {} & {} & {} \\
                \bottomrule
              \end{tabular}
              \caption{Translation table for data in Lab4PokerData.csv}
            \end{table}
            To make it easier to understand the data, every column were assigned with a name. Following 
            table shows an example of the labels used.
            \begin{table}[H]
              \centering
              \label{tbl:task1f}
              \begin{tabular}{llll}
                \toprule
                p1\_hand\_category & p1\_hand\_rank &  p1\_hand\_value & p1\_action\_$n$ \\ p1\_raise\_$n$ & p1\_money\_left & p1\_spent\_per\_hand\_value & p1\_final\_action \\
                \midrule
                p2\_hand\_category & p2\_hand\_rank &  p2\_hand\_value & p2\_action\_$n$ \\ p2\_raise\_$n$ & p2\_money\_left & p2\_spent\_per\_hand\_value & p2\_final\_action \\
                \bottomrule
              \end{tabular}
              \caption{Labels of each column in Lab4PokerData.csv, $n$ is for each round.}
            \end{table}
            The added attributes are \textit{pX\_money\_left, pX\_spent\_per\_hand\_value \& pX\_final\_action}.
            "\textit{pX\_money\_left}" is calculated by summing how much the player bet each round and subtract it from $400$.
            "\textit{pX\_spent\_per\_hand\_value}" is derived from dividing how much a player spent on that round with the given hand value.
            
          \subsection{b)}
            \begin{table}[h]  
              \label{tbl:task2b}
              \centering
              \begin{tabular}{lrr}
                \toprule
                Classifier              &   Score ($R^2$)&  Time (s) \\
                \midrule
                KNN, $k=5$                    &   0.75 &  0.000464 \\
                SVC, $kernel=linear,~C=0.025$ &   0.95 &  0.017687 \\
                Decision Tree, $max\_depth=5$ &   1.00 &  0.000918 \\
                \bottomrule
              \end{tabular}
              \caption{Performance of three different classification algorithms on the Poker Data set.}
            \end{table}
            Here we get some really interesting results, the Decision Tree Classifier manages to score 
            $1.00$ which means it got every prediction correct. The SVC algorithm did also perform really good with
            a score of $0.95$. The KNN-Classifier did not perform as good as the other two algorithms with a score of
            only $0.75$, but it was the fastest among the three.
          
          \subsection{c)}
            \begin{table}[h]
              \label{tbl:task2c}
              \centering
              \begin{tabular}{cccc}
                \toprule
                Classifier &  Neighbors &       Distance Measure &           Score (Mean/Std)\\
                \midrule
                KNN &          3 &              Manhattan &  0.75 (+/-0.22) \\
                KNN &          5 &              Manhattan &  0.80 (+/-0.25) \\
                KNN &          7 &              Manhattan &  0.70 (+/-0.34) \\
                KNN &          9 &              Manhattan &  0.60 (+/-0.29) \\
                KNN &         11 &              Manhattan &  0.45 (+/-0.12) \\
                KNN &          3 &              Euclidean &  0.75 (+/-0.22) \\
                KNN &          5 &              Euclidean &  0.70 (+/-0.30) \\
                KNN &          7 &              Euclidean &  0.57 (+/-0.25) \\
                KNN &          9 &              Euclidean &  0.40 (+/-0.19) \\
                KNN &         11 &              Euclidean &  0.42 (+/-0.12) \\
                KNN &          3 &  Minkowski p=$2^{1.5}$ &  0.70 (+/-0.20) \\
                KNN &          5 &  Minkowski p=$2^{1.5}$ &  0.65 (+/-0.33) \\
                KNN &          7 &  Minkowski p=$2^{1.5}$ &  0.60 (+/-0.19) \\
                KNN &          9 &  Minkowski p=$2^{1.5}$ &  0.42 (+/-0.25) \\
                KNN &         11 &  Minkowski p=$2^{1.5}$ &  0.42 (+/-0.12) \\
                \bottomrule
              \end{tabular}
              \caption{Cross-Validation score of KNN-Classifier with different amount of neighbors and distance algorithms}
            \end{table}
            When trying different parameters for the KNN-Classifier we can tell by the results in Table \ref{tbl:task2c} that the Manhattan Distance as distance measure and with $k=5$ perform
            best. We can also tell that the score drops rapidly with $k > 5$ for all the different distance measures tested.
\end{document}