\documentclass{article}
\usepackage[utf8]{inputenc}
\usepackage{booktabs}
\usepackage{float}
\usepackage[a4paper, width=150mm, top=25mm, bottom=25mm, bindingoffset=6mm]{geometry} 


\title{AI Lab 2}
\author{Karl-Johan Djervbrant}

% Hide section numbering
\makeatletter
\def\@seccntformat#1{%
  \expandafter\ifx\csname c@#1\endcsname\c@section\else
  \csname the#1\endcsname\quad
  \fi}
\makeatother

\begin{document}

\maketitle

\section{Task 1 Path Planning}
\subsection{a) Unified Search Algorithms}
In this task we were supposed to implement three different types of search algorithms.
\begin{enumerate}
    \item Random Agent
    \item Breadth First Agent
    \item Depth First Agent 
\end{enumerate}
The random agent works by simply adding a random weight to the node between 1 and 10, then sort the list of all nodes by the weight and remove the first node in the list.\\
Breadth First and Depth first are very similar in how they are implemented, it's simply a difference in how one removes a node from the list. In Depth first, the last element in
the list is removed first whereas in Breadth first you remove the first element.
\subsubsection{Performance}
On a set of 100 runs on a 100 by 100 map, this is the mean for each algorithm.
\begin{figure}[h]
    \center
    \begin{tabular}{llllr}
        \toprule
        {Mean of 100 runs} & {Breadth First Search} & {Depth First Search} & {Random Search} \\
        \midrule
        nodes      &  4725.61 & 5857.09 & 5445.28\\
        pathLength &   144.84 & 4413.16 & 292.60 \\
        \bottomrule
    \end{tabular}
\end{figure}

Out of the three Unified Search Algorithms we can se that out of 100 runs, the Breadth First Search algorithm performs best, though none of the algorithms are particularly good 
when it come to the amount of expanded nodes.
\subsection{a) Informed Search Algorithms}

\subsection{b) USA vs ISA on map with obstacle}

\subsection{b) A* with heuristic}

\subsection{What I learned in Task 1}

\newpage
\section{Task 2 Poker Bidding}
In this task we had to implement three different kinds of search algorithms to beat
an opponent player in a game of poker. The implemented algorithms were a Random-, Breadth First-
and a Greedy Search algorithm. The key used when sorting the list of states for the Greedy Search algorithm
were the \texttt{nn\_current\_hand} which is a measure of how many hands the dealer have handed out in this state.
I use this key since we sought to find the state were the agent win more than 100 coins from the opponent in the least
amount of hands.
\subsection{The implemented agents}
\subsubsection{Performance}
\begin{figure}[H]
    \center
    \begin{tabular}{llllr}
        \toprule
        {100 Games} & {Random Search} & {Breadth First Search} & {Greedy Search}  \\
        \midrule
        stack         &   306.50 &     400.30 &   402.10 \\ 
        nNodes        &  9102.80 &  135871.28 &  4209.28 \\
        depth         &    13.24 &      10.36 &    10.64 \\
        nHands        &     7.22 &       2.30 &     2.10 \\
        opponentStack &   293.50 &     199.70 &   197.90 \\
        \bottomrule
    \end{tabular}
       
\end{figure}
After playing 100 games of poker we can see that the Greedy Search algorithm performs the best
out of the three algorithms, with fewest number of expanded nodes. It also wins over the opponent
for most of the times with only two hands. The search depth on the other hand is on average a bit
larger compared to the BFS algorithm. Random search performs the worst out of all three, not only
does it expand many nodes, it also doesn't win very much.

\subsection{What I learned in Task 2}
In this task I increased my understanding of reading and understanding code someone else had written,
and also the importance of sorting your data on the correct key to increase the performance of a search algorithm.
\end{document}